\chapter{Conclusioni}
Concludiamo questa relazione sul progetto con delle brevi conclusioni sulle conseguenze che il sistema Numsynth ha mostrato nei vari esperimenti descritti nel Capitolo~\ref{cap:esperimenti} e riassunti nel Capitolo~\ref{cap:risultati}.

\myskip

Numsynth permette di introdurre il ragionamento numerico in un ILP system. Questo nelle situazioni dove il tempo è fondamentale per apprendere qualche relazione, è una cosa fondamentale. Un altro vantaggio è quello che, quando impara le regole corrette, lo fa isolando bene i predicati che sono effettivamente necessari alla spiegazione, senza includere in essa predicati che possono essere evitati.

\myskip

Il sistema è estremamente sensibile alla presenza dei corretti predicati. La sensazione è che riesca a imparare buone regole se e solo se sono stati inclusi nel representation language i predicati che spiegano effettivamente i fallimenti delle tracce. Se questi predicati non ci sono, o ne mancano anche solo alcuni, il sistema non riesce a trovare una regola che spieghi, anche solo parzialmente, il problema. Se per tracce semplici e brevi questi predicati possono essere trovati a mano, quando abbiamo molti eventi e molte combinazioni, allora il problema può diventare non risolvibile.

\myskip

Un ulteriore limite di Numsynth, forse anche maggiore del precedente, è la complessità. Il sistema non riesce ad arrivare alla soluzione se il numero di tracce è troppo elevato. Inoltre è ancora più sensibile al numero di tracce positive, cioè al numero di tracce che deve spiegare: anche se il numero di tracce totali rimane lo stesso, aumentando quelle che falliscono si può non arrivare alla soluzione.