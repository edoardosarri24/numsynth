\chapter{Esperimenti}
Dopo aver eseguito una prima fase di analisi nel Capitolo~\ref{cap:analisi}, adesso andiamo a descrivere gli esperimenti e i risultati a cui questi portano. Di rilievo sarà la ì descrizione del representation bias.

\myskip

Quando si esegue un esperimento è necessario, oltre ad avere a disposizione il file di log \textit{trace.log}, definire solo il representation bias. È stato infatti pensato uno script, che dovrà essere modificato in base all'esperimento e trovarsi nella cartella, che legge il file di log e definisce i file tipici di Popper (e quindi di Numsynth) \textit{bk.pl} e \textit{exs.pl} necessarie per il sistema.

Una volta che abbiamo definito una cartella per l'esperimento contenente un file \textit{bias.pl} e uno di log \textit{trace.log}, possiamo lanciare l'esperimento dalla root directory con \texttt{./numsynth.sh <folder-name>}.

%%%%%%%%%%%%%%%%%%%%%%%%%%%%%%%%%%%%%%%%%
\section{Primo approccio}
Il primo approccio tentato è stato il più semplice possibile: abbiamo inserito all'interno del representation bias solo quei predicati che rappresentano ciò che i log del simulatore di taskset mostra. Questi nello specifico sono: rilascio e completmento di un task; inizio e fine dell'esecuzione di un chunk. Nella cartella \textit{my-experiments} questo esperimento è \href{https://github.com/edoardosarri24/numsynth/tree/main/my-experiments/1-start/}{start}.

In questo modo stiamo chiedendo al sistema di spiegarci il fallimento o il successo della traccia su cui facciamo inferenza solamente tramite questi predicati. Sarà poi Numsynth a cercare di trovare delle relazioni, anche numeriche, tra chunk e task per coprire il maggior numero di esempi positivi e non coprire nessuno di quelli positivi.

\subsection{Modello}
\label{subsec:mod-primo-approccio}
Ogni traccia in questo esperimento ha una durata di massimo 100ms e sono state generate 100 tracce, di cui 76 hanno terminato con successo e 24 fallendo.

\myskip

Il modello usato per questo esperimento, essendo il primo, è stato semplice e forse banale. È stato usato Rate Monotonic (RM) come algorimto di scheduling e il taskset ha queste caratteristriche:
\begin{itemize}
    \item Task1: periodo 3ms; deadline relativa 3ms; chunk1 con execution time campionato da un ConstantSampler con parametro 1ms; chunk2 con execution time campionato da un UniformSampler con parametri 0.5ms e 1.1.
    \item Task2: periodo 5ms; deadline relativa 5ms; chunk1 con execution time campionato da un ConstantSampler con parametro 1ms.
\end{itemize}

Volevamo ottenere un taskset che fosse facile da comprendere. Una traccia dovrebbe avere sempre successo tranne quando il secondo chunk del primo task (i.e., chunk1.2) campio un valore superiore a 1. Essendo sotto RM , in questo caso al task2 non rimarebbe abbastanza tempo per eseguire entro la propria deadline.

\subsection{Risultati}
I risultati, che possiamo osservare nel file \href{https://github.com/edoardosarri24/numsynth/tree/main/my-experiments/1-start/result.txt}{result.txt}, mostrano la semplicità del modello scelto per questo esperimento.
\begin{itemize}
    \item Il tempo di addestramento è di circa 25s.
    \item La precision e la recall sono entrambe massime e pari a 1. Questo significa che sono stati coperti tutti gli esempi negativi e nessuno di quelli positivi.
    \item La soluzione è composta da quattro letterali: \texttt{failure(A):- mult(E,100,F), \allowbreak finish(A,F,D,C),finish(A,E,D,C)}.
\end{itemize}

\myskip

La soluzione che il sistema ha trovato ci porta a dei ragionamenti che ne evidenziano le problematiche.

In primo luogo la generalizzazione è molto bassa. Il sistema ha trovato una regola che è il massimo dei True Positive (TP) e il massimo dei True Negative (TN); probabilmente su log più complessi questa regola non riuscirà a prevedere il fallimento della traccia in modo corretto.

Dalla regola appresa possiamo vedere come il sistema classifica una traccia come corretta in base alla sua lunghezza. Nello specifico infatti la traccia ha successo se esiste un chunk \textit{D.C} che termina la sua esecuzione sia al tempo \textit{E} che al tempo \textit{F}, dove $F=E\cdot100$. Siccome 100 è il massimo valore di una simulazione (i.e., il tempo massimo in una traccia), se una traccia fallisce allora sicuramente non arriva al tempo 100. Per confermare questo è stato fatto l'esperimento \href{https://github.com/edoardosarri24/numsynth/tree/main/my-experiments/2-start250/}{start250}, identico al precedente tranne che per la lunghezza delle tracce che è stata portata fino a 250ms. Tra le varie informazioni del file \href{https://github.com/edoardosarri24/numsynth/tree/main/my-experiments/2-start250/result.txt}{result.txt} possiamo osservare quanto previsto: la regola imparata è \texttt{failure(A):- finish(A,E,C,F),finish(A,B,C,F),mult(B,247,E)}, che ci conferma quanto ipotizzato.

%%%%%%%%%%%%%%%%%%%%%%%%%%%%%%%%%%%%%%%%%
\section{Inversione dei positivi e negativi}
Per cercare di impedire al sistema di utilizzare la lunghezza delle tracce come criterio per valutare la corretterzza di una traccia, sono stati invertiti gli esempi positivi e negativi, lasciando tutto il resto come descritto nella Sezione~\ref{subsec:mod-primo-approccio}, ed è stato così prodotto l'esperimento \href{https://github.com/edoardosarri24/numsynth/tree/main/my-experiments/3-posNeg-inversion}{posNeg-inversion}.

In questo modo il sistema dovrà cercare di spiegare come mai una traccia fallisce e non perché ha successo; non dovrebbe quindi usare la lunghezza della traccia come parametro fondamentale, visto che il fallimento potrebbe avvenire all'inizio della traccia, ma anche un attimo prima del tempo massimo della simulazione.

\subsection{Risultati}
Il risultato di questo esperimento mostra come i predicati descritti fino a questo momento nel representation bias (i.e., il file \textit{bias.pl}) non sono sufficienti.

Come si spiega nel paper~\cite{numsynth}, per ogni task di ricerca è predisposto un timer di 600s che permette al sistema di entrare in un loop. Questo timer in questo caso viene superato e la ricerca si arresta durante la ricerca di una soluzione con cinque letterali.

\myskip

La migliore soluzione trovata in questo esperimento ha quindi 4 letterali ed è \\ \texttt{failure(A):- execute(A,C,B,F),mult(C,56,E),finish(A,E,B,F)}. Il fatto che il sistema utilizzi ancora le moltiplicazioni per spiegare il fallimento (o successo) di una traccia è un chiaro segnale che il representation bias attuale non è sufficiente.

%%%%%%%%%%%%%%%%%%%%%%%%%%%%%%%%%%%%%%%%%
\section{Without multiplications}
\label{sec:wo-mult}
Per cercare di evitare che il sistema utilizzi le moltiplicazioni per spiegare il fallimento o il successo di una traccia, gli è stat levata la possibilità di utilizzare le moltiplicazioni nell'esperimento \href{https://github.com/edoardosarri24/numsynth/tree/main/my-experiments/4-wo-mult/}{wo-mult}.

Il modello utilizzato è sempre quello del primo esperimento descritto nella Sezione~\ref{subsec:mod-primo-approccio}. L'unica modifica che è stata portata è quella di eliminare le moltiplicazioni dal representation bias.

\subsection{Risultati}
La prima cosa interessante è notare come il sistema non abbia trovato una soluzione di dimensione quattro, cosa che invece succede nell'omonimo esperimento con la moltiplicazione come predicato utilizzabile.

La seconda cosa che è chiara è che dopo un certo numero di letterali, il sistema non riesce a gestire l'elevata ricorsione.

\myskip

Il risultato finale, che possiamo osservare nel file \href{https://github.com/edoardosarri24/numsynth/tree/main/my-experiments/4-wo-mult/result.txt}{result.txt}, ci fornisce due spunti molto importanti:
\begin{itemize}
    \item Senza moltiplicazioni non si si riesce a spiegare il fallimento con una regola che abbia sia un'alta precision che un'alta recall.
    \item Senza predicati aggiuntivi, l'unico modo di spiegare il successo di una traccia in modo efficace (i.e., con un'alta precision e recall) è tramite la sua lunghezza.
\end{itemize}

%%%%%%%%%%%%%%%%%%%%%%%%%%%%%%%%%%%%%%%%%
\section{Magic Popper}
Come abbiamo detto nella Sezione~\ref{sec:numsynth}, Numsynth estende Popper con i predicati numerici. Oltre a questo permette anche di utilizzare le caratteristiche di MagicPopper~\cite{hocquette2023learning}, cioè permette di far si che in una soluzione non compaiano solamente valori presenti nella background knowledge, ma anche valori inventati.

Da questa idea nasce l'esperimento \href{https://github.com/edoardosarri24/numsynth/tree/main/my-experiments/5-magic/}{magic}. La configurazione è esattamente la stessa di quello senza la moltiplicazione presentato nella Sezione~\ref{sec:wo-mult}; nel representation bias è stato aggiunto \texttt{magic\_value\_type(int)} che permette a Numsynth di inventare valori per il tipo intero, cioè per il tempo.

\subsection{Risultati}
I risultati sono esattamente quelli attesi. La regola trovata è \texttt{failure(A):- \\ release(A,95000,D),release(A,0,D)} e ci mostra quanto il sistema si basi sulla lunghezza delle regole per classificarle.

Essa infatti ci dice che, perchè una traccia abbia successo, deve esistere un task \textit{D} che viene rilasciato sia al tempo 0 che al tempo 95000, che praticamente è quasi il tempo massimo che una traccia può durare (100ms).

Ovviamente questo non è quanto vogliamo e ci dice ancora di più che i predicati attuali non sono sufficienti.