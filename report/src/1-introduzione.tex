\chapter{Introduzione}
Questo progetto è stato svolto durante la laurea magistrale in Ingegneria Informatica per il corso di \textit{Explainable Artificial Intelligence}.

L'idea di \textit{Prediction in Data-Driven System} nasce dalla situazione in cui all'interno di un'azienda abbiamo un sistema di cui non è noto il modello, cioè non conosciamo i task che lo compongono e le loro caratteristiche, ma abbiamo solo i dati prodotti dal logging. In questo scenario, se abbiamo la necessità di valutare quale task potrebbe portare a dei problemi, e quindi quale dovrebbe essere modificato o studiato più approfonditamente, non possiamo fare un'analisi statica sul modello ma possiamo solo considerare i dati.

Vorremmo avere un metodo che, dato uno storico di file di log, possa prevedere se una traccia di cui non conosciamo l'esito (e.g., la traccia di log attuale che il sistema ci fornisce) avrà un fallimento e da quale task questo fallimento sarà causato.

%%%%%%%%%%%%%%%%%%%%%%%%%%%%%%%%%%%%%%%%%
\section{Scenario real-time}
Nel nostro contesto specifico i dati sono prodotti da un sistema real-time, cioè un sistema i cui task devono terminare entro una definita deadline. In questo contesto il fallimento di una traccia sarà quindi dovuto al superamento della deadline di un task.

Essendo in uno scenario simulato e non avendo un sistema reale, come prima cosa ci è servito un generatore di dati che ci fornisse il dataset su cui applicare le nostre idee. Per generare i dati è stato quindi usato un altro mio progetto che ci fornisce un simulatore per lo scheduling di task real-time. Tale simulatore, presente nella repo del mio GitHub \href{https://github.com/edoardosarri24/real-time-scheduling-simulator}{real-time-scheduling-simulator}, è in grado di generare un dataset di tracce dato un modello di taskset e un algoritmo di scheduling.

%%%%%%%%%%%%%%%%%%%%%%%%%%%%%%%%%%%%%%%%%
\section{ILP System}
\label{sec:numsynth}
Per imparare la motivazione del possibile fallimento di una traccia è stato usato un ILP (Inductive Logic Programming) system.

Questo ci permette di utilizzare la logica per definire una groundtruth, e dei predicati per spiegare il fallimento di una traccia.

\myskip

Il tool utilizzato per eseguire il nostro studio è stato \href{https://github.com/celinehocquette/numsynth-aaai23}{Numsynth}. Si tratta di un'estensione di Popper che permette anche il ragionamento numerico. È stato deciso di utilizzare questo sistema per due motivi: essendo basato su Popper, ha dietro un sistema noto e solido; può eseguire valutazioni su valori numerici, cosa fondamentale visto che i file di log sono basati su tempi in cui accadono eventi; utilizza il Learning From Failure (LFE) che permette, nonostante la complessità del ragionamento numerico, di limitare lo spazio di ricerca della soluzione.

Numsynth permette di specificare dei predicati che sono particolari e che ci potrebbero essere utili nel seguito:
\begin{itemize}
    \item \texttt{direction(predicato,(dir1,dir2,...))}: \\
        Permette di specificare quali valori devono essere noti e quali saranno restituiti dal predicato. Le direzioni possibili sono: \textit{in} definisce che il valore deve essere noto, cioè già introdotto in precedenza nella regola; \textit{out} definisce che il valore sarà restituito dal predicato.
    \item \texttt{max\_vars(n)}: \\
        Definisce il vincolo sulla complessità delle regole, specificando il numero massimo di variabili che una regola può contenere.
    \item \texttt{max\_body(n)}: \\
        Definisce il vincolo sulla complessità delle regole, specificando il numero massimo di letterali che il corpo di una regola può avere.
    \item \texttt{magic\_value\_type(type)}: \\
        Definisce il tipo di valori che il sistema può inventare. Solitamente infatti i valori (anche quelli numerici) che compaiono nella soluzione sono solo quelli presenti nella background knowledge.
    \item \texttt{bounds(predicato,arg\_pos,(min, max))}: \\
        Definisce i limiti in cui ricercare un valore numerico. Serve per diminuire lo spazio di ricerca e la complessità del training. Di default questo valore è 100.
\end{itemize}